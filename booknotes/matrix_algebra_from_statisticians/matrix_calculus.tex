\documentclass[a4paper]{article}
\usepackage[english]{babel}
\usepackage{color,graphicx,amsmath,hyphenat}
\setlength{\topmargin}{0in}
\setlength{\oddsidemargin}{0in}
\setlength{\textwidth}{6.25in}
\usepackage{xcolor}
\setlength{\textheight}{9in}
\setcounter{MaxMatrixCols}{20}
\pagestyle{empty}

% some shortcuts
\newcommand{\ea}{\textit{et al. }} 
\newcommand{\eg}{\textit{e.g. }} 
\newcommand{\ie}{\textit{i.e. }} 
\newcommand{\la}{\langle}
\newcommand{\ra}{\rangle}
\newcommand{\cg}{\color{gray}}
\newcommand{\tr}{\text{tr}}
\newcommand{\rank}{\text{rank}}
\newcommand{\adj}{\text{adj}}
\newcommand{\diag}{\text{diag}}
\newcommand{\fs}{\footnotesize}
\newcommand{\mb}{\mathbf}

\begin{document}
\pagestyle{empty}
  
  
\section*{Summary of Matrix Calculus}

\section*{15.1 Definitions, Notation and Preliminaries}
\begin{itemize}
\item Derivative of scalar-valued function $f$ with input $\mathbf{x}=(x_1,\hdots,x_m)$
\begin{itemize}
\setlength{\parindent}{0mm}
\item Interior point: 	$\{\mathbf{x} \in \mathcal{R}^{m\times 1}: ||\mathbf{x}-\mathbf{c}||<r\}$ {\cg 	for some pos. const $r$} \\
{\cg\fs Applies to matrices too:\\
$\{\mathbf{X}\in\mathcal{R}^{m\times n} : ||\mathbf{X}-\mathbf{C}||<r\}$
}
\item \textbf{Def:} \textit{j$^{th}$ (first) partial derivative (scalar-value function with vector input $\mathbf{x}=(x_1,\hdots,x_m)^{T}$} \\
$D_j f(\mathbf{c})$ denotes the $j$th part. deriv. of $f$ at $\mathbf{c}$: \\
$D f_j(\mathbf{c})=\lim\limits_{t\rightarrow 0}=\frac{f(\mathbf{c}+t\mathbf{u}_j)-f(\mathbf{c})}{t}$ {\cg where $\mathbf{u}_j$ is the $j$th row of identity mx.} \\
Alternative notation: $\frac{\partial f(\mathbf{x})}{\partial x_j}$ \\
Alternative notation: At times, it may be more convenient to reshape vector $x$ as matrix $\mathbf{X}$ and denote its partial derivative wrt element $x_{ij}$ as $\frac{\partial f(\mathbf{X})}{x_{ij}}$. The way we treat this derivative depends on whether the elements of $\mathbf{X}$ are dependent (\eg symmetric matrix) or independent (Sec. 15.1.f)

\item \textbf{Def:} \textit{Vector of partial derivatives}
$\mathbf{D} f(\mathbf{c})$ denotes vector of all part. derivs of $f$ at $\mathbf{c}$:\\
$\mathbf{D}f(\mathbf{c}) = (D_1 f(\mathbf{c}), D_2 f(\mathbf{c}), \hdots, D_m f(\mathbf{c}))'$\\
Alternative notation: $\frac{\partial f(\mathbf{x})}{\partial \mathbf{x}'}$
\item \textbf{Def:} $(\mathbf{D}f)'$  is called \textit{gradient vector} \\
Alternative notation: $\frac{\partial f(\mathbf{x})}{\partial \mathbf{x}}$

\item \textbf{Def: } \textit{continuously differentiable}: \\
Function $f$ with domain $S\in \mathcal{R}^{m\times 1}$ is continuously differentiable at the interior pt $\mathbf{c}\in S$ if $D_1f(\mathbf{x}),\hdots, D_mf(\mathbf{x})$ exist and are continuous at every pt in some neighbourhood of $\mathbf{c}$.\\
In this case the following holds: $\lim\limits_{\mathbf{x}\to\mathbf{c}} \frac{f(\mathbf{x})-[f(\mathbf{c})+\mathbf{D}f(\mathbf{c})(\mathbf{x-c})]}{||\mathbf{x-c}||}$

\item \textbf{Def:} \textit{ij}$^{th}$ \textit{(second) partial derivative}
$D_{ij}^2f(\mathbf{x)}$ \\
Alternative notation: $\frac{\partial^2 f(\mathbf{x})}{\partial x_i x_j}$
\item \textbf{Def:} \textit{Hessian Matrix} $\mathbf{H}f$\\
An $m\times m$ matrix whose \textit{ij}th element is $D_{ij}^2f(\mathbf{x)}$
\end{itemize}
\item Derivative of vector-valued fn $\mathbf{f}=(f_1,\hdots,f_p)'$ where each $f_i$ takes input $\mathbf{x}=(x_1,\hdots,x_m)'$.
\begin{itemize}
	\item $D_jf_s(\mathbf{c})$: $j$th partial derivative of $f_s$
	\item $D_j\mathbf{f(c)}$ is $p\times 1$ vector $D_j\mathbf{f(c)}=[D_jf_1(\mathbf{c}), \hdots, D_jf_p(\mathbf{c})]'$ \\
	Alternative notation $\frac{\partial f(\mathbf{x})}{\partial\mathbf{x}'}$ (row vector, $1\times p$ \--- see Sec 15.1.c \#287) \\
	Alternative notation $\frac{\partial f(\mathbf{x})}{\partial\mathbf{x}}$ (column vector, $p\times 1$)
	\item $\mathbf{Df}$ is $p\times m$ matrix: $\mathbf{Df(c)}=[D_1\mathbf{f(c)}, \hdots, D_p\mathbf{f(c)}]$\\
	$\mathbf{Df}$ is called \textit{Jacobian} of $\mathbf{f}$ and it's the matrix whose $sj$th element is $D_jf_s$. \\
	$(\mathbf{Df})'$ is called the \textit{gradient (matrix)} of $\mathbf{f}$. \\ 
	Alternative notation to Jacobian: $\frac{\partial\mathbf{f(x)}}{\partial \mathbf{x}'}$ and it's the matrix whose $sj$th element is $\frac{\partial f_s(\mathbf{x})}{\partial x_j}$
\end{itemize}
\item Derivative of matrix of functions $F={f_{st}}$ where $\mathbf{F}$ is $p\times q$
\begin{itemize}
\item It's preferable to keep the $j$th partial derivatives of $\mathbf{F}$ as a separate $p\times q$ matrix, denoted as :
$\frac{\partial\mb{F}(\mb{x})}{\partial x_j}$ or $D_j\mb{F(x)}$
\end{itemize}
\end{itemize}

\section*{15.2 Differentiation of Scalar-valued Functions}
\begin{itemize}
\setlength{\parindent}{0mm}
\item Lem 15.2.1 \--- If $f(\mathbf{x})$ does not vary wrt $x_j$ at $\mathbf{c}$ then $D_jf(\mathbf{c})=0$
\item Lem 15.2.2 \--- Let $l,h,r$ be functions defined as:
$l=af+bg$, $h=fg$ and $r=f/g$. The rules of derivative for single-variable calculus function apply for the $j$th partial derivative of $l,h,g$.
\end{itemize}
\section*{15.3 Differentiation of Linear and Quadratic Forms}
\begin{itemize}
\setlength{\parindent}{0mm}
\item Let $\mathbf{a}=(a_1,\hdots,a_m)$ be constant (or fn of $\mb{x}$ that is invariant wrt $x_j$), and $\mb{A}$ be an $m\times m$ constant matrix (or matrix of functions invariant wrt $x_j$). Then:
\begin{itemize}
\setlength{\parindent}{0mm}
\item $\frac{\partial  \mb{a'x}}{\partial x_j}=a_j$ (see \#294)\\
{\cg\fs The core idea is to see that $\frac{\partial x_i}{\partial x_j}=\begin{cases}1 &\text{ if }i=j\\0& \text{ else}\end{cases}$}
\item $\frac{\partial  \mb{a'x}}{\partial \mathbf{x}}=\mb{a}$ or $\frac{\partial  \mb{a'x}}{\partial \mathbf{x}'}=\mb{a}'$
\item $\frac{\partial \mb{x'Ax}}{\partial x_j} = \sum\limits_{i=1}^m a_{ij}x_i+\sum\limits_{k=1}^{m} a_{jk}x_k$ (see \#295)\\ 
{\cg\fs The core idea is again a similar piecewise (4-case) function as above}
\item $\frac{\partial \mb{x'Ax}}{\partial \mb{x}} = (\mathbf{A+A'})\mathbf{x}$ \---- if $\mb{A}$ symmetric then $\frac{\partial \mb{x'Ax}}{\partial \mb{x}} = 2\mathbf{Ax}$
\item $\frac{\partial^2 \mb{x'Ax}}{\partial x_s x_j} = a_{sj}+a_{js}$ (see \#295)
\item $\frac{\partial^2 \mb{x'Ax}}{\partial \mb{x}^2} = (\mathbf{A+A'})$ \---- if $\mb{A}$ symmetric then $\frac{\partial^2 \mb{x'Ax}}{\partial \mb{x}^2} = 2\mathbf{A}$
\end{itemize}
\end{itemize}

\section*{15.4 Differentiation of Matrix Sums, Products and Transposes}
Now we consider to function-valued matrices $\mb{F,G,H}$
\begin{itemize}
\item $\frac{\partial a\mb{F}+b\mb{G}}{\partial x_i} = a\frac{\partial \mb{F}}{\partial x_i}+b\frac{\partial \mb{G}}{\partial x_i}$
\item $\frac{\partial \mb{FG}}{\partial x_j} = \mb{F}\frac{\partial \mb{G}}{\partial x_j}+\frac{\partial \mb{F}}{\partial x_j}\mb{G}$
\item $\frac{\partial \mb{FGH}}{\partial x_j} = \mb{FG}\frac{\partial \mb{H}}{\partial x_j}+\mb{F}\frac{\partial \mb{G}}{\partial x_j}\mb{H}+\mb{FG}\frac{\partial \mb{H}}{\partial x_j}$
\item if $g$ is fn of $\mathbf{x}: \frac{\partial g\mb{F}}{\partial x_i} = \frac{\partial g}{\partial x_i}\mb{F}+g\frac{\partial \mb{F}}{\partial x_i}$
\end{itemize}

\section*{Section 15.5 -- 15.7 }
\begin{itemize}
\item 15.5 Differentiation of Vector/Matrix $\mb{x,X}$ wrt its Elements $x_j,x_{ij}$
\begin{itemize}
\item $\frac{\partial \mathbf{x}}{\partial x_i} = \mb{u}_j$
\item if $\mb{X}$ matrix of independent variables $x_ij$:\\
$\frac{\partial \mathbf{X}}{\partial x_{ij}} = 	\mathbf{u}_i\mb{u}_j'$
\item if $\mb{X}$ symmetric matrix: \\
$\frac{\partial \mb{X}}{\partial x_{ij}} = \mb{u}_i\mb{u}_j'+\mb{u}_j	\mb{u}_i'$
\item The above are derived by constructing piecewise functions for $\frac{\partial x_{st}}{\partial x_{ij}}$ by considering all cases for $s,i,j,t$ (\ie when they are equal, unequal etc.)

$\frac{\partial a\mb{F}+b\mb{G}}{\partial x_i} = a\frac{\partial \mb{F}}{\partial x_i}+b\frac{\partial \mb{G}}{\partial x_i}$
\item $\frac{\partial \mb{FG}}{\partial x_j} = \mb{F}\frac{\partial \mb{G}}{\partial x_j}+\frac{\partial \mb{F}}{\partial x_j}\mb{G}$
\item $\frac{\partial \mb{FGH}}{\partial x_j} = \mb{FG}\frac{\partial \mb{H}}{\partial x_j}+\mb{F}\frac{\partial \mb{G}}{\partial x_j}\mb{H}+\mb{FG}\frac{\partial \mb{H}}{\partial x_j}$
\item if $g$ is fn of $\mathbf{x}: \frac{\partial g\mb{F}}{\partial x_i} = \frac{\partial g}{\partial x_i}\mb{F}+g\frac{\partial \mb{F}}{\partial x_i}$
\end{itemize}
\item 15.6 Differentiation of a Trace of a Matrix {\cg -- again trace is critical for more complicated differentiations}
\begin{itemize}
\item Some trace properties:
\begin{itemize}
\item $\tr(AB)=\tr(BA)$
\item $\tr(A+B)=\tr(A)+\tr(B)$
\end{itemize}
\item $\frac{\partial \tr{\mb{F}}}{\partial x_j} = \tr\left(\frac{\partial\mb{F}}{\partial x_j}\right)$
\item $\frac{\partial (\mb{AX})}{\partial x_{ij}}=a_{ji}$
\item $\frac{\partial (\mb{AX})}{\partial \mb{X}}=A'$
\item if $\mb{X}$ is symmetric:\\
$\frac{\partial (\mb{AX})}{\partial \mb{X}}=\mb{A}+\mb{A}'-\diag(a_{11},a_{22}, \hdots, a_{mm})$ \\
\item Regardless whether $\mb{X}$ is symmetric or not, $\frac{\partial \tr{\mb{X}}}{\partial \mb{X}} = \mb{I}$
\end{itemize}
\item 15.7 The Chain Rule
\begin{itemize}
\item Thm 15.7.1: Let $\mb{h} = \{h_i\}$ be an $n\times 1$ vector of functions of variables $\mb{x}=(x_1,\hdots,x_m)$. Let $g$ be a scalar-valued function of a vector of $\mb{y}=(y_1,\hdots,y_n)$. Define $f(\mb{x}) = g[\mb{h(x)}]$. Then, the $j$th partial derivative of $f$: \\
 $D_j f(\mb{c}) = \sum_{i=1}^n D_i g[\mb{h}(c)]D_j h_i(\mb{c}) = \mb{D}g[\mb{h(c)}]D_j\mb{h(c)}$
 \item Alternative notation: $\frac{\partial f}{\partial x_j} = \sum_{i=i}^n \frac{\partial g}{\partial y_i }\frac{\partial h_i}{\partial  x_j} = \frac{\partial g}{\partial \mb{y}'}\frac{\partial \mb{h}}{\partial x_j}$
\item The vector of all first partial derivatives:\\
$\mb{D}f(\mb{c}) = \sum_{i=1}^n D_i g[\mb{h(c)}]\mb{D}h_i (\mb{c}) = \mb{D}g[\mb{h(c)}]\mb{D(h(c))}$ \\
Alternative notation: $\frac{\partial f}{\partial \mb{x}'} = \sum_{i=i}^n \frac{\partial g}{\partial y_i }\frac{\partial h_i}{\partial  \mb{x}'} = \frac{\partial g}{\partial \mb{y}'}\frac{\partial \mb{h}}{\partial \mb{x}'}$
 \item For vector-valued function $\mb{f}$: \\
 $D_j \mb{f(c)} = \sum\limits_{i=1}^n D_i\mb{g[h(c)]}D_j h_i(\mb{c}) = \mb{Dg[h(c)]}D_j\mb{h(c)}$
\item For all partial derivatives of $\mb{f}$: \\
 $\mb{Df(c)}=\sum\limits_{i=1}^n D_i \mb{g[h(c)]}\mb{D}h_i(\mb{c})=\mb{Dg[h(c)]Dh(c)}$\\
 Alternative notation: $\frac{\partial \mb{f}}{\partial \mb{x}'} = \sum_{i=i}^n \frac{\partial \mb{g}}{\partial y_i }\frac{\partial h_i}{\partial  \mb{x}'} = \frac{\partial \mb{g}}{\partial \mb{y}'}\frac{\partial \mb{h}}{\partial \mb{x}'}$
\end{itemize}
\end{itemize}

\section*{Section 15.8 -- Derivs of Determinants, Inverses, Adjugates and Generalized inverses}
\begin{itemize}
\item $\frac{\partial \det(\mb{X})}{\partial x_{ij} }=\xi_{ij}$ where $\xi_{ij}$ is the $ij$th cofactor of $\mb{X}$ and $\mb{X}$ is a matrix of variables (and not functions)
\item $\frac{\partial \mb{X}}{\partial \mb{X}} = [\adj(\mb{X})]'$ where $\adj()$ is adjugate
\item $\frac{\partial \det(\mb{F})}{\partial x_{j} }=\tr\left[\adj(\mb{F})\frac{\partial \mb{F}}{\partial x_j}\right]\stackrel{(a)}{=}|\mb{F}|\tr\left(\mb{F}^{-1}\frac{\partial\mb{F}}{\partial x_j}\right)$ where $\mb{F}$ is a matrix of functions (a) follows only if $\mb{F}$ is nonsingular and differentiable.
\item $\frac{\partial \log\det(\mb{X})}{\partial x_{ij}}=\tr({\mb{X}^{-1}\mb{u}_i\mb{u}_j'})=\mb{u}_j'\mb{X}^{-1}\mb{u}_i=y_{ji}$ where $y_{ji}$ is $ji$th element of $\mb{X}^{-1}$
\item $\frac{\partial \log \det (\mb{F}}{\partial x_j } = \frac{1}{|\mb{F}|} \frac{\partial\det(\mb{F})}{\partial x_j} = \frac{1}{|\mb{F}|}\tr\left[\adj(\mb{F})\frac{\partial \mb{F}}{ \partial x_j}\right]=\tr\left(\mb{F}^{-1}\frac{\partial \mb{F}}{\partial x_j}\right)$
\item if $\mb{X}$ symmetric matrix of variables:
\begin{itemize}
\item $\frac{\partial\det(X)}{\partial\mb{X}}=2\adj(\mb{X})-\diag(\xi_{11}, \xi_{22}, \hdots, \xi_{mm})$
\item $\frac{\partial\log\det(X)}{\partial\mb{X}}=2\adj(\mb{X}^{-1})-\diag(y_{11}, y_{22}, \hdots, y_{mm})$ where $y_{ij}$ are elements of $\mb{X}^{-1}$
\end{itemize}
\item $\frac{\partial \adj(\mb{F})}{\partial x_j}=\frac{\partial|\mb{F}|}{\partial x_j}\mb{F}^{-1}+|\mb{F}|\frac{\partial \mb{F}^{-1}}{\partial x_j}=\hdots$ (can be completed with info above)
\end{itemize}
\section*{Section 15.9 \-- Second-order derivatives of Determinants and Inverses}
\section*{Differentiation of Generalized Inverses}
The section considers derivatives of generalized inverses of possibly singular or non-square matrix of functions $\mb{F}$. The idea is to re-order the rows and cols of $\mb{F}$ and then partition the new matrix so the leading principal submatrix is nonsingular. 

\paragraph{Summary of Theorem 15.10.1}

Let $\mb{P,Q}$ be permutation matrices that yield matrix $\mb{B}=\mb{PFQ}$ such that the leading principal submatrix $\mb{B}_{11}$ in partitioning $\mb{B}=\left(\begin{matrix}
\mb{B}_{11} & \mb{B}_{12} \\ \mb{B}_{21} & \mb{B}_{22}
\end{matrix}\right)$ is an $r\times r$ nonsingular matrix where $r=\rank(\mb{F})$. Then, there exists a generalized inverse of $\mb{G}$ of $\mb{F}$ such that: $\mb{G} = \mb{Q\begin{bmatrix}
\mb{B}^{-1}_{11} & 0 \\ 0 & 0
\end{bmatrix}P}$ and its derivative is:
\begin{equation}
\frac{\partial \mb{G}}{\partial x_j} = -\mb{Q} \begin{bmatrix}
\mb{B}^{-1}_{11}(\partial \mb{B}_{11}/\partial x_j)\mb{B}^{-1}_{11} & 0 \\ 0 & 0
\end{bmatrix}\mb{P} = -\mb{G}\frac{\partial \mb{F}}{\partial x_j}\mb{G}
\end{equation}

\clearpage
\newpage

\bibliographystyle{IEEEtran}
\bibliography{report}


\end{document}










