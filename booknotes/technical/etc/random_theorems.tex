\documentclass{article}
\usepackage{amsmath}
\usepackage{hyperref}
\usepackage{xcolor}
\usepackage{amsfonts}
\usepackage{graphicx}
\topmargin=-0.65in    % Make letterhead sftart about 1 inch from top of page
\textheight=9.10in    % text height can be bigger for a longer letter
\oddsidemargin=-0.1in % leftmargin is 1 inch
\textwidth=6.7in   % textwidth of 6.5in leaves 1 inch for right margin


% some shortcuts
\newcommand{\ea}{\textit{et al. }} 
\newcommand{\eg}{\textit{e.g. }} 
\newcommand{\ie}{\textit{i.e. }} 
\newcommand{\la}{\langle}
\newcommand{\ra}{\rangle}
\newcommand{\cg}{\color{gray}}
\newcommand{\fs}{\footnotesize}
\setlength{\parindent}{0mm}

%%%%%%%%%%%%%%%%%%%%%%%%%%%%%%%%%%%%%%%%%%%%%%%%%%%%%%%%%%%%%%%%%%%%%%%%
%%%%%%%%%%%%%%%%%%%%%%%%%%%%%%%%%%%%%%%%%%%%%%%%%%%%%%%%%%%%%%%%%%%%%%%%
%%%%%%%%%%%%%%%%%%%%%%%%%%%%%%%%%%%%%%%%%%%%%%%%%%%%%%%%%%%%%%%%%%%%%%%%
%%%%%%%%%%%%%%%%%%%%%%%%%%%%%%%%%%%%%%%%%%%%%%%%%%%%%%%%%%%%%%%%%%%%%%%%
\begin{document}


%%%%%%%%%%%%%%%%%%%%%%%%%%%%%%%%%%%%%%%%%%%%%%%%%%%%%%%%%%%%%%%%%%%%%%%%
%%%%%%%%%%%%%%%%%%%%%%%%%%%%%%% OUTLINE %%%%%%%%%%%%%%%%%%%%%%%%%%%%%%%%
%%%%%%%%%%%%%%%%%%%%%%%%%%%%%%%%%%%%%%%%%%%%%%%%%%%%%%%%%%%%%%%%%%%%%%%%
\title{\bf Random equations and identities}
\maketitle
\section*{Chapter 5}

$\mathcal{R}(AB) = R(A)$ if $B$ is an invertible matrix.

Riesz basis condition  implies the invertibility of the Gram matrix (footnote of FSP page 94).

Finite geom series: 
\begin{equation}
\sum\limits_{k=a}^{b}t^k=\begin{cases} \frac{t^a-t^{b+1}}{1-t} & \text{for } t\neq 1 \\ b-a+1 &
 \text{for } t=1
 \end{cases}
\end{equation}

\section*{Approximations}

\begin{itemize}
\item \textbf{Edgeworth series} are used to approximate a probability distribution in terms of its cumulants.
\item \textbf{Gram-Charlier series} very similar to Edgeworth series. 
\end{itemize}



% https://www.youtube.com/watch?v=VXwXkME9uWU&list=PLMn2aW3wpAtOqo0g0OnHndXB1LnYBeMaX&index=1
\end{document}

