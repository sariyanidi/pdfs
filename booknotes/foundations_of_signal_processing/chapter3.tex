\documentclass{article}
\usepackage{lipsum}
\usepackage{amsmath}
\usepackage{amssymb}
\usepackage{hyperref}
\usepackage{xcolor}
\usepackage{amsfonts}
\topmargin=-0.65in    % Make letterhead sftart about 1 inch from top of page
\textheight=9.10in    % text height can be bigger for a longer letter
\oddsidemargin=-0.1in % leftmargin is 1 inch
\textwidth=6.7in   % textwidth of 6.5in leaves 1 inch for right margin

% some shortcuts
\newcommand{\ea}{\textit{et al. }} 
\newcommand{\eg}{\textit{e.g. }} 
\newcommand{\ie}{\textit{i.e. }} 
\newcommand{\la}{\langle}
\newcommand{\ra}{\rangle}
\newcommand{\cg}{\color{gray}}
\newcommand{\fs}{\footnotesize}
\setlength{\parindent}{0mm}

%%%%%%%%%%%%%%%%%%%%%%%%%%%%%%%%%%%%%%%%%%%%%%%%%%%%%%%%%%%%%%%%%%%%%%%%
%%%%%%%%%%%%%%%%%%%%%%%%%%%%%%%%%%%%%%%%%%%%%%%%%%%%%%%%%%%%%%%%%%%%%%%%
%%%%%%%%%%%%%%%%%%%%%%%%%%%%%%%%%%%%%%%%%%%%%%%%%%%%%%%%%%%%%%%%%%%%%%%%
%%%%%%%%%%%%%%%%%%%%%%%%%%%%%%%%%%%%%%%%%%%%%%%%%%%%%%%%%%%%%%%%%%%%%%%%
\begin{document}


%%%%%%%%%%%%%%%%%%%%%%%%%%%%%%%%%%%%%%%%%%%%%%%%%%%%%%%%%%%%%%%%%%%%%%%%
%%%%%%%%%%%%%%%%%%%%%%%%%%%%%%% OUTLINE %%%%%%%%%%%%%%%%%%%%%%%%%%%%%%%%
%%%%%%%%%%%%%%%%%%%%%%%%%%%%%%%%%%%%%%%%%%%%%%%%%%%%%%%%%%%%%%%%%%%%%%%%
\title{\bf Outline of Foundations of Signal Processing}
\author{Evangelos Sariyanidi}
\maketitle
\section*{Chapter 3}
Discrete(-time) Fourier Transform, Z-Transform, Stochastic Processes

\begin{itemize}

\item Discrete-time (DT) Systems {\cg \textit{Operators} having sequences as their inputs and outputs}\\
{\cg\fs Some types \#195: \color{gray} Linear, Memoryless, Causal, Shift-invariant, Stable (BIBO). {\bf We focus on Linear\&Shift-Invariant}}
\item Basic systems {\color{gray} Shift, modulator, accumulator, averaging operator, maximum}
\item Table 3.3 (\#203) !important! illustrates types of systems and their properties!
\item Difference equations (DE): many systems can be described via DEs. \\
{\color{gray} \footnotesize
DE systems are NOT linear or shift-invariant UNLESS the initial conditions are zero (Appendix 3.A.2)
}
\item Linear Shift-invariant Systems (LSISs)
\begin{itemize}
	\item Impulse response {\color{gray} $h$: the of an LSI sys for input $\delta$}
	\item Convolution {\color{gray} $(Hx)_n=(h\ast x)_n=\sum_{k\in\mathbb{Z}}x_kh_{n-k}=\sum_{k\in\mathbb{Z}}h_kx_{n-k}$} \\
	{\color{gray}\footnotesize
	Connection to inner product: $(h\ast x)_n=\sum_{x\in\mathbb{Z}}x_kh_{n-k}=\la x_k, h_{n-k}^{*} \ra$\\
	Commutativity, Associativity\\
	Deterministic autocorrelation (AC): $a_n = x_n\ast_n x^{*}_{-n}$ \\
	Shifting: $x_n\ast_n\delta_{n-k}=x_{n-k}$	\\
	IMPORTANT! Most properties above depend on the sums -- must be abs. converging.
	}
	\item Filter: {\color{gray}Impulse response of a system}; Filtering: {\color{gray} convolving a seq with impulse response}.
	\item THM 3.8: (BIBO) Stability {\color{gray}An LSIS is BIBO-stable iff its impulse response is abs. summable. \\
	Limiting in $\ell^2(\mathbb{Z})$ avoids convergence issues \#209}
	\item Circular conv: {\color{gray} applies to finite-length ($N$) seqs, extended circularly	}
 \\
	{\color{gray} \footnotesize
	$(h\circledast x)_n=\sum_{k=0}^{N-1}h_kx{(n-k)\mod N}$}
	\item THM 3.10 - Equiv of circular and linear conv: \\
	{\color{gray} If length of $x$ is $M$ and of $h$ is $L$, linear and circular conv is equiv if $N\ge L+M-1$}
\end{itemize}
\item Discrete-time Fourier Transform (DTFT) {\color{gray} is a cont fn of $\omega$; $X(e^{j\omega})=\sum_{n\in\mathbb{Z}}x_n {\upsilon^{-1}_{n}}$ where $\upsilon_n=e^{j\omega n}$} \\
Inverse DTFT: {\cg $x_n=\frac{1}{2\pi}\int\limits_{-\pi}^{\pi}X(e^{j\omega n})e^{j\omega n}d\omega$}\\
{\color{gray} \footnotesize
IMPORTANT! Why FT?: "Simply b/c they are based on eigensequences of LSIS":\\
$(H\upsilon)_n=\sum_{k\in\mathbb{Z}}h_k e^{j\omega (n-k)} = \sum_{k\in\mathbb{Z}}h_k e^{-j\omega k} e^{j\omega n} = \lambda_{\omega} e^{j\omega n} = \lambda_{\omega} \upsilon_n$ \---- an eigendecomposition
}
\begin{itemize}
\item Convergence in the mean square  \\
{\color{gray} \footnotesize if sequence not abs. convergent, we can pursue convergence in mean square $(\mathcal{L}^2)$. There may be points that don't converge, but we may remove them through Dirac delta fn \#220}.
\item Properties of DTFT: (Table 3.4 \#222) \\
{\color{gray}\footnotesize 
Linearity, shift in time, shift in freq, scaling in time, time reversal, differentiation, moment computation, conv ($x\ast y=XY$), circular convolution in freq, deterministic AC ($a_n=|X(e^{j\omega})|^2$), deterministic crosscorelation (CC) ($c_{x,y,n}=X(e^{j\omega})Y^{*}(e^{j\omega})$)}
\item Parseval equality \#226 (energy conservation): {\color{gray} $||X||^2=2\pi ||x||^2$; $X=Fx$, $F:\ell^2(\mathbb{Z})\rightarrow \mathcal{L}^2([-\pi,\pi])$ }
\item Freq Resp of filter: {\color{gray} The DTFT of a filter (\ie impulse resp of an LSIS) $h$: $H(e^{j\omega})=\sum_{n\in \mathbb{Z}} h_n e^{-j\omega n}$}\\
{\color{gray}\footnotesize
We often write magnitude and phase resp separately: $H(e^{j\omega})=|H(e^{j\omega})|e^{j\arg(H(e^{j\omega}))}$
\\
{Zero-phase Filter: Filter that has real freq resp\\
Generalised Linear-Phase Filter: $H(e^{j\omega})=|H(e^{j\omega})|e^{\alpha j+\beta}$ \\
Linear-Phase Filter: $H(e^{j\omega})=|H(e^{j\omega})|e^{\alpha j}$ }
}
\item Ideal Filter: {\cg Unrealizable filter, needs infinite support but able to pass requested frequencies exactly}
{\cg\fs Ideal lowpass filt: $\sqrt{\frac{\omega_0}{2\pi}}\text{sinc}(\frac{1}{2}\omega_0 n) \rightarrow \begin{cases}\sqrt{2\pi/\omega_0} & |\omega|\le \frac{1}{2}\omega_0 \\ 0 & other\end{cases}$ \\
Ideal $N$th-band filter: $\frac{1}{\sqrt{N}}\text{sinc}(\frac{\pi n}{N}) \rightarrow \begin{cases}\sqrt{N} & |\omega|\le \frac{\pi}{N} \\ 0 & other\end{cases}$
}
\item Finite Impulse Response (FIR) Filters: {\cg realizable filters} \\
\item Linear-Phase Filters: {\cg Real-valued FIR filters that are symmetric or antisymmetric}
\item Allpass filters: \\
{\cg \fs
i) Energy conservation: $||y||^2=||x||^2$ \#231
ii) Ortonormal set: Shifts of $h$, $\{h_{n-k}\}_{k\in\mathbb{Z}}$, form an orthonormal set\\
iii) Orthonormal basis: $\{\phi_k\}_{k\in\mathbb{Z}}$ where $\phi_{k,n}=h_{n-k}$ form an orthonormal basis for $\ell^2(\mathbb{Z})$}

\end{itemize}
\item $z$-transform {\cg DTFT cool but assumes convergence. $z$-transforms relaxes this through region of convergence (ROC) concept; $X(z)|_{z=e^{j\omega}}=X(e^{j\omega}) \implies $ DTFT is a special case of $z$-transform. }
\begin{itemize}
	\item Defn: {\cg $X(z) = \sum_{n\in\mathbb{Z}}x_nz^{-n}$ and $\text{ROC} = \{z:|X(z)|< \infty \}$ where $\upsilon_n=z^n=r^{n}e^{j\omega n}$}
	\item Properties similar to DTFT; Table 3.6 \#243
	\item Rational $z$-transforms {\cg important class, $H(z)=\frac{B(z)}{A(z)}$}
	\item THM 3.13: Rational AC {\cg A rational fn A(z) is the deterministic $AC$ of a stable real sequence $x$, iff ... \#245}
	\item Corollary 3.14: Spectral Factorization \#247 {\cg A rational z-trans $A(z)$ is the deterministic AC of a stable real sequence $x$ iff it can be factored as $A(z)=X(z)X(z^{-1})$}
	\item THM 3.15 (BIBO stability with rational fns) {\cg A causal LSIS is BIBO-stble iff the poles of its Transfer Function (TF) are} inside the unit circle.
	\item !important! ROC can't contain any poles \#238, \#246
\end{itemize}
	\item Discrete Fourier Trans (DFT) {\cg While DTFT is $F_{\text{DTFT}}:\ell^2(\mathbb{Z})\rightarrow \mathcal{L}^2([-\pi,\pi])$, DFT is $F_{\text{DFT}}:\ell^2(\mathbb{Z})\rightarrow\ell^2(\mathbb{Z})$ }\\
	{\cg\fs
	DFT: $X_k = (Fx)_k = \sum\limits_{n=0}^{N-1}x_nW_N^{kn};$\\
	$W_N^{nk}=e^{-j kn 2\pi/N}; \upsilon_n=e^{j\omega n}$ \\
	Inverse DFT: $x_n = \frac{1}{N}(F^{*}X)_n =\frac{1}{N}\sum\limits_{n=0}^{N_1}X_k W_N^{-kn}$ 
	}
	\begin{itemize}
		\item Again, is eigenseq \#252: {\cg$(H\upsilon)_n=(h\circledast \upsilon)_n=\sum\limits_{i=0}^{N-1}\upsilon_{(n-i)\mod N}h_i =\sum\limits_{i=0}^{N-1}W_N^{ki)}h_iW_N^{-kn}=\sum\limits_{i=0}^{N-1}\lambda_k \upsilon_n$}
		\item $k$: Discrete Frequency
		\item Relation to DFT \#255: {\cg $X(e^{j\omega})|_{\omega=\omega_k}=X_k$; $\omega_k=\frac{k2\pi}{N}$}
		\item Properties similar to DTFT, Table 3.7 \#256
		\item Modulation: {\cg A shift $k_0$ in frequency \#257}
		\item Frequency response of filters: {\cg DFT of a N-dim filter (impulse resp of an LSIS) $h$ is frequency response:} \\
		{\cg\fs $H_k=\sum\limits_{n=0}^{N-1}h_n W_N^{kn}$\\
		Again, we separate as magnitude and phase response $H_k=|H_k|e^{j\arg(H_k)}$ 
		}
		\item !Beware! DFT analysis of infinite-long sequences can be misleading!\\
		{\cg Misleading allpass behaviour \#260\\
		Misleading linear-phase
		}
		\item !Beware! On periodic seqs, windowing becomes crucial to avoid SPURIOUS high frequencies
	\end{itemize}
	\item ... Multirate seqs and systems (Sec 3.7, \#264--\#285) ...
	\begin{itemize}
		\item Upsampling and downsampling not commutative in general; {\cg only when upsampling and downsampling rates $M,N$ are relatively prime (they have no common factors)}
		\item Hermitian transposition equals time reversal (\#277)
		\item Quadrature mirror formula (3.217 \#277) or \textit{power complementarity}\\
		{\cg\fs This is central in the design of orthonormal filter banks} 
	\end{itemize}
	
	\item DT Stochastic Processes (SP)\\
	{\cg A countably infinite collection of jointly distributed RVs $\{\hdots, x_0, x_1, \hdots \}$\\
	We study systems that act deterministically on Random signals.}
	\begin{itemize}
		\item 2nd order statistics: {\cg mean, var, std, AC, CC}
		\item For iid, $\mu_{x,n}=\mu_x,\sigma_{x,n}=\sigma_x$ etc. \#286
		\item Stationarity \#287 {\cg Generalizes the iid prop by allowing dependence bw RVs}\\
		{\cg\fs An SP is stationary if the joint distros of $(x_{n_0},x_{n_1},\hdots,x_{n_L})$ and $(x_{n_0+k},\hdots,x_{n_L+k})$ are identical for finite $\{n_0,\hdots,n_L\}$}
		\item Wide-sense SP (WSS) {\cg More relaxed than stationary processes}\\
		{\cg\fs When $\mu_{x,n}=\mu_x$, AC $a_{x,n,k}=a_{x,k}$. Two SPs $x,y$ are jointly WSS when $c_{x,y,n,k}=c_{x,y,k}$}
		\item White Noise: {\cg An SP such that $\mu_{x,n}=0$ and $\sigma_{x,n}=\sigma_x$, $a_{x,n,k}=\sigma_x^2\delta_k$}
		\item Orthogonality (Tab 3.10 \#301) {\cg $c_{x,y,k}=\text{E}[x_n y^{*}_{n-k}]=0;$ in frequency: $C_{x,y}(e^{j\omega})=0$}		
		\item Whitening (decorrelation) {\cg Processing that results in a white noise proc}
		\item System analysis {\cg Props of sys output $y$ when SP $x$ is input}\\
		{\cg\fs$\mu_{y,n}=\mu_y$, $a_{y,n,k}=a_{y,k}$, $c_{x,y,n,k}=c_{x,y,k}$}
		\item Autoregressive Moving-average (ARMA) Process {\cg Output of a BIBO-stable causal LSIS when input is white noise $x$.} \\
		{\cg\fs One generative model is: $y_n=\sum\limits_{k=0}^M b_k x_{n-k}+\sum\limits_{k=1}^N a_k y_{n-k}$}
		\item Moving Average (MA) Process {\cg when $a_i=0 \forall i$ in the eq above}
		\item Autoregressive (AR) Process {\cg when $b_i=0$ !for $i>0$!}
	\end{itemize}
	
\end{itemize}


The definitions of power and energy can be very confusing for discrete vs continuous or deterministic seqs vs SPs. Tab 3.8 (copied below) \#293  clarifies:

\begin{table}[h!]
\centering
\small
\begin{tabular}{l l} \hline
\textbf{Deterministic  sequences} & \textbf{WSS discrete-time stochastic processes} \\\hline\hline
Energy spectral density & Power spectral density\\
$A(e^{j\omega})=|X(e^{j\omega})|^2$ & $A(e^{j\omega})=\sum\limits_{k\in\mathbb{Z}}\text{E}[x_nx^{*}_{n-k}e^{-j\omega k}]$ \\ 
 & \\
 Energy & Power \\
 $E=\frac{1}{2\pi}\int\limits_{-\pi}^{\pi}A(e^{j\omega})d\omega$ &
 $P=\frac{1}{2\pi}\int\limits_{-\pi}^{\pi}A(e^{j\omega})d\omega$ \\
 $E = a_0 = \sum\limits_{n\in\mathbb{Z}}|x_n|^2$  & $P=a_0=\text{E}[|x_n|^2]$ \\ \hline
\end{tabular}
\end{table}



% https://www.youtube.com/watch?v=VXwXkME9uWU&list=PLMn2aW3wpAtOqo0g0OnHndXB1LnYBeMaX&index=1
\end{document}
