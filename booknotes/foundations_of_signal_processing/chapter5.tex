\documentclass{article}
\usepackage{lipsum}
\usepackage{amsmath}
\usepackage{hyperref}
\usepackage{xcolor}
\usepackage{amsfonts}
\usepackage{graphicx}
\topmargin=-0.65in    % Make letterhead sftart about 1 inch from top of page
\textheight=9.10in    % text height can be bigger for a longer letter
\oddsidemargin=-0.1in % leftmargin is 1 inch
\textwidth=6.7in   % textwidth of 6.5in leaves 1 inch for right margin

% some shortcuts
\newcommand{\ea}{\textit{et al. }} 
\newcommand{\eg}{\textit{e.g. }} 
\newcommand{\ie}{\textit{i.e. }} 
\newcommand{\la}{\langle}
\newcommand{\ra}{\rangle}
\newcommand{\cg}{\color{gray}}
\newcommand{\fs}{\footnotesize}

%%%%%%%%%%%%%%%%%%%%%%%%%%%%%%%%%%%%%%%%%%%%%%%%%%%%%%%%%%%%%%%%%%%%%%%%
%%%%%%%%%%%%%%%%%%%%%%%%%%%%%%%%%%%%%%%%%%%%%%%%%%%%%%%%%%%%%%%%%%%%%%%%
%%%%%%%%%%%%%%%%%%%%%%%%%%%%%%%%%%%%%%%%%%%%%%%%%%%%%%%%%%%%%%%%%%%%%%%%
%%%%%%%%%%%%%%%%%%%%%%%%%%%%%%%%%%%%%%%%%%%%%%%%%%%%%%%%%%%%%%%%%%%%%%%%
\begin{document}
\author{Outline by: Evangelos Sariyanidi}

%%%%%%%%%%%%%%%%%%%%%%%%%%%%%%%%%%%%%%%%%%%%%%%%%%%%%%%%%%%%%%%%%%%%%%%%
%%%%%%%%%%%%%%%%%%%%%%%%%%%%%%% OUTLINE %%%%%%%%%%%%%%%%%%%%%%%%%%%%%%%%
%%%%%%%%%%%%%%%%%%%%%%%%%%%%%%%%%%%%%%%%%%%%%%%%%%%%%%%%%%%%%%%%%%%%%%%%
\title{\bf Outline of Foundations of Signal Processing}
\maketitle
\section*{Chapter 5 -- Sampling and Interpolation}

The chapter considers how to move from discrete to continuous spaces (and vice versa), which were studied in earlier chapters. This is essential for many studies including wavelet analysis.\\

Throughout the chapter two things are placed to focus: Interpolation followed by sampling (IFS) and sampling followed by interpolation (SFI). These two are discussed for three types of inputs: i) Continuous functions, ii) sequences and iii) periodic functions. For each type of input the discussion is repeated for a) orthogonal vectors and b) nonorthogonal vectors (often biorthogonal pairs of bases).

\subsection*{Finite-dimensional Sequences}
\begin{itemize}
\setlength{\parindent}{0mm}
\item Orthogonal vectors
\begin{itemize}
	\item Thm 5.1 \--- Recovery from orthogonal vecs (\#423) \\
{\cg \fs Let $\Phi^{*}:\mathbb{C}^M\rightarrow \mathbb{C}^N$ be sampling operator and $\Phi:\mathbb{C}^N\rightarrow \mathbb{C}^M$ interpolation operator. \\
	Approximated fn is: $\hat{x} = Px = \Phi\Phi^{*}x$, the best approximation of $x$ in $S$. \\
	If rows of $\Phi^{*}$ orthonormal, error is orthogonal to $S=\mathcal{R}(\Phi)$, \ie $x-\hat{x}\perp S$} 
\end{itemize}
\item Non-orthogonal vectors 
\begin{itemize}
	\item We have sampling op $\tilde{\Phi}^{*}$ and interpolation op $\Phi$. \\
	{\cg\fs
	Define $\tilde{S}=\mathcal{N}(\tilde{\Phi}^{*})^{\perp}=\text{span}(\{\tilde{\phi_k}\}_{k=0}^{N-1})$ and \\
	$S=\mathcal{R}(\Phi)$}
	\item Def \-- Consistency (\#428): {\cg When $\tilde{\Phi}^{*}\Phi=I$, ops $\tilde{\Phi}^{*}, \Phi$ are called \textit{consistent}}
	\item Def \-- Ideally Matched (\#427): {\cg When $S=\tilde{S}$.}
	\item Thm 5.2 \-- Recovery from non-orthogonal vectors (\#428) \\
	{\cg\fs
	Similar to Thm 5.1 but we need consistency to satisfy $x-\hat{x}\perp \tilde{S}$. \\
	If ideal matching also exists, approximation $\hat{x}=Px=\Phi\tilde{\Phi}^*x$ is best approximation, and also $S=\tilde{S}$
	}
\end{itemize}
\end{itemize}

\subsection*{Sequences and Functions}
Now we consider (infinite-dimensional) sequences (Sec 5.3) and functions (Sec 5.4). Of particular importance will be the \textit{bandlimited} sequences, which we'll be able to represent with a finite number of coefficients. The abstract space $S$ above will be typically replaced by the space of bandlimited sequences $\text{BL}[\omega_1,\omega_2]$.

\begin{itemize}
\item Sequences
\begin{itemize}
\item Def \-- Shift-invariant Subspace of $\ell^2(\mathbb{Z})$ \\
{\cg\fs
$S\subset \ell^2(\mathbb{Z})$ is shift-inv. subspace \textit{with respect to a shift} $L\in\mathbb{Z}^+$ when $x{_n\in S}$ implies $x{n-kL}\in S$ for every $k\in\mathbb{Z}$.}
\item Def \-- Generator of $S$ \\
{\cg\fs
$s\in\ell^2(\mathbb{Z})$ is called a \textit{generator} of $S$ when $S=\overline{\text{span}}(\{s_{n-kL}\}_{k\in\mathbb{Z}})$
}
\item Thm 5.4 \-- Recovery from sequences (very similar to Thm 5.1 above)
\item Def \-- Bandlimited sequence (\#437) {\cg(otherwise called full-band sequence)}\\
{\cg\fs
A seq $x$ is called \textit{bandlimited} when there is $\omega_0\in[0,2\pi)$ s.t. its DTFT $X$ satisfies $X(e^{j\omega}=0$ for all $|\omega|\in (\frac{1}{2}\omega_0,\pi]$
}
\item Def \-- Bandwidth {\cg the smallest $\omega_0$ for bandlimited sequence.}
\item Def \-- Subspace of Bandlimited seqs (\#438) \\
{\cg\fs Set of sequences in $\ell^2(\mathbb{Z})$ with bandwidth at most $\omega_0$ is \textit{a closed subspace} denoted $\text{BL}[-\frac{1}{2}\omega_0,\frac{1}{2}\omega_0]$. }
\item Thm 5.7 (\#439) Projection to BL spaces \\
{\cg\fs The best approximation of $x$ on $\text{BL}[\pi/N,\pi/N]$ involves ideal LP filters (\ie trunctation of spectrum $x$ to $[-\pi/n,\pi/N]$):\\
$\hat{x}_n=\frac{1}{\sqrt{N}}\sum\limits_{k\in\mathbb{Z}}y_k \text{sinc}(\frac{\pi}{N}(n-kN))$, $n\in\mathbb{Z}$ where \\
$y_k=\frac{1}{\sqrt{N}}\sum\limits_{n\in\mathbb{Z}}x_n\text{sinc}(\frac{\pi}{N}(n-kN))$}
\item Thm 5.8 \-- Sampling theorem for seqs (\#440) \\
{\cg\fs
If $x$ in $\text{BL}[\-pi/N,\pi/N]$: \\
$x_n=\sum\limits_{k\in\mathbb{Z}}x_{kN}\text{sinc}(\frac{\pi}{N}(n-kN))$
}
\item Def \-- Aliasing (\#440): {\cg When the component of $x$ at freq $\omega$ affects a component $\tilde{\omega}\neq \omega$ (i.e. assumes \--aliases\-- the role of the component)}
\item Thm 5.9 \-- Recovery for sequences, non-orthogonal \\
{\cg\fs Counterpart of Thm 5.2 for infinite-dim sequences.}
\end{itemize}
\item Functions
\begin{itemize}
	\item Very similar definitions for {\cg bandwidth (\#452), subspace of bandlimited functions (\#453), aliasing (\#460) and theorem of projection to bandlimited subspace (\#454).}
	\item Thm 5.15 -- Sampling Theorem (cornerstone theorem) \\
	{\cg\fs If $x\in BL[-pi/T, pi/T]$, then $x(t)=\sum\limits_{n\in \mathbb{Z}}x(nT) sinc(\frac{\pi}{T}(t-nT))$}
	\item Thm 5.16: Continuous convolution via discrete operators (\#462) \\
	{\cg\fs Let $x\in BL[\frac{-\pi/T}{\pi/T}]$, and $\tilde{h}_n=\langle h(t), sinc(\frac{\pi}{T}(n-T))_t\rangle$. Let sampled signal be $\tilde{x}_n=\sqrt{T}x(nT)$ and discrete conv output be $\tilde{y}=\tilde{x}\ast\tilde{x}$. Then: \\
	$y(t) = \sqrt{T}\sum_{n\in\mathbb{Z}}\tilde{y}_n sinc(\frac{\pi}{T}(t-nT))$}
\end{itemize}
\item Periodic Functions
\begin{itemize}
	\item Similar definitions for bandlimited and bandwidth ($k_0$ s.t. $X_k=0$ for $k>k_0$, see \#481), subspace of bandlimited fns (\#482), projection to bandlimited subspace (Thm 5.25, \#484), and sampling thm (Thm 5.26 \#489).
	\item Dirichlet Kernel (\#482-\#483): The counterpart of the ideal LP filter (with $\text{sinc}$)for periodic functions. 
	{\cg\fs
	$d(t)=\sum\limits_{k=-K}^K e^{j(2\pi/T)}kt$ \\
	* Acts like a LP filter and forms a basis for $\text{BL}[\cdot,\cdot]$ space. (\#483)\\
	* Plays the role of interpolating filter sinc in sampling Thm 5.26 (\#489)}
\end{itemize}
\item Projection onto convex sets via Papoulis-Gerhberg algorithm. \\
{\cg\fs A nice algorithm for signal reconstruction.\\
Assumption: $x$ is bandlimited perioduc function in $\mathcal{L}^2([0,1])$ where a part for $t\in(\alpha,\beta)$ is unobserved/missing.\\
Then, we know $x$ belongs to two sets: $S_1=BL\{\hdots\}$ and $S_2=\mathcal{L}^2([0,1])$.\\
We find the full signal by exploiting these two set memberships in an alternating and iterative manner (\#492)\\
Discrete version from in \#494}


\end{itemize}

%* The rate at which the approximation error $\hat{x}_T-x$ goes to zero for as $T\rightarrow 0$ is an important indicator of the approximation power of the interpolation scheme.

%* Def: shift-invariant space and generator (\#430)
%* 5.3.2 Shift-invariant subspaces are subspaces of band-limited spaces!!!
%*Def: spectral replicas and base specrum \#441
%* interpolation and sampling are adjoints of each other	
%* #460 when freqs not centred at zero, we can do bandpass postfiltering to %recover signal
%* #462 -> Thm 5.16 emulating CT op
%* \# 464 only an ideal filter can afford to sample at critical sampling %rateramfi
%* Ex 5.26 CRITICAL how to find ideally matched postfilter so that %$S=\tilde{S}$

%* Ex 5.27 nice imperfect reconstruction  of triangle wave
%* Why can we skip prefiltering for BL Functions?
%* Projection onto convex sets!

%** follow the general structure of the chapter. 

%errata \#434 shouldn't it be $\hat{x}$ 
%errata \#446 shouldn't $\hat{x} = Px$
%errata \#528 shouldn't "first five" be "first four"?

	
% https://www.youtube.com/watch?v=VXwXkME9uWU&list=PLMn2aW3wpAtOqo0g0OnHndXB1LnYBeMaX&index=1
\end{document}


