\documentclass{article}
\usepackage{lipsum}
\usepackage{amsmath}
\usepackage{hyperref}
\usepackage{xcolor}
\usepackage{amsfonts}
\usepackage{graphicx}
\topmargin=-0.65in    % Make letterhead sftart about 1 inch from top of page
\textheight=9.10in    % text height can be bigger for a longer letter
\oddsidemargin=-0.1in % leftmargin is 1 inch
\textwidth=6.7in   % textwidth of 6.5in leaves 1 inch for right margin

% some shortcuts
\newcommand{\ea}{\textit{et al. }} 
\newcommand{\eg}{\textit{e.g. }} 
\newcommand{\ie}{\textit{i.e. }} 
\newcommand{\la}{\langle}
\newcommand{\ra}{\rangle}
\newcommand{\cg}{\color{gray}}
\newcommand{\fs}{\footnotesize}


\newtheorem{theorem}{Theorem}[section]
\newtheorem{lemma}[theorem]{Lemma}
\newtheorem{proposition}[theorem]{Proposition}
\newtheorem{corollary}[theorem]{Corollary}

\newcommand{\qed}{\nobreak \ifvmode \relax \else
      \ifdim\lastskip<1.5em \hskip-\lastskip
      \hskip1.5em plus0em minus0.5em \fi \nobreak
      \vrule height0.75em width0.5em depth0.25em\fi}

\newenvironment{proof}[1][Proof]{\begin{trivlist}
\item[\hskip \labelsep {\bfseries #1}]}{\end{trivlist}}
\newenvironment{definition}[1][Definition]{\begin{trivlist}
\item[\hskip \labelsep {\bfseries #1}]}{\end{trivlist}}
\newenvironment{example}[1][Example]{\begin{trivlist}
\item[\hskip \labelsep {\bfseries #1}]}{\end{trivlist}}
\newenvironment{remark}[1][Remark]{\begin{trivlist}
\item[\hskip \labelsep {\bfseries #1}]}{\end{trivlist}}

\setlength{\parindent}{0mm}

%%%%%%%%%%%%%%%%%%%%%%%%%%%%%%%%%%%%%%%%%%%%%%%%%%%%%%%%%%%%%%%%%%%%%%%%
%%%%%%%%%%%%%%%%%%%%%%%%%%%%%%%%%%%%%%%%%%%%%%%%%%%%%%%%%%%%%%%%%%%%%%%%
%%%%%%%%%%%%%%%%%%%%%%%%%%%%%%%%%%%%%%%%%%%%%%%%%%%%%%%%%%%%%%%%%%%%%%%%
%%%%%%%%%%%%%%%%%%%%%%%%%%%%%%%%%%%%%%%%%%%%%%%%%%%%%%%%%%%%%%%%%%%%%%%%
\begin{document}
\section{Week 1}
\begin{lemma} (Opposite Triangle Inequality) Let $v,w \in V$. Then $||v-w||\geq \big| ||v||-||w|| \big|$
\end{lemma}
\begin{proof}
Break down into two: (a) $||v-w||\geq ||v||-||w||$ and (b) $||v-w||\geq -(||v||-||w||)$. To prove (a):
\begin{equation}
||v|| = ||v-w+w|| \leq ||v-w||+||w|| \implies ||v-w||\geq ||v||-||w||
\end{equation}
Part (b) can be proved easily with the same logic. \qed
\end{proof}

\begin{lemma} (subspace) A nonempty subset $W \subset V$ is a subspace iff
\begin{equation}
\alpha v+ \beta w \in W\,\, \forall v,w \in W, \alpha, \beta \in \mathbb{C}
\end{equation}
\end{lemma}

\begin{theorem}
(Supremum attained by continuous fns on closed intervals)
Let $[a,b]$ be a closed, bounded interval. Then any cont fn $f:[a,b]\rightarrow \mathbb{R}$ attains its supremum, \ie $\exists x_0 \in [a,b]$ such that $f(x_o) = sup{f(x) | x \in [a,b]}$.
\end{theorem}
\begin{proof}
See Ex 2.1.4 in \cite{christensen10}.
\end{proof}

\paragraph{The opposite triangle inequality}

\bibliographystyle{plain}
\bibliography{refs}



% https://www.youtube.com/watch?v=VXwXkME9uWU&list=PLMn2aW3wpAtOqo0g0OnHndXB1LnYBeMaX&index=1
\end{document}


